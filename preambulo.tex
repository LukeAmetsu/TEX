% ---
% PACOTES
% ---

% ---
% Pacotes fundamentais 
% ---
\usepackage{lmodern}			% Usa a fonte Latin Modern
\usepackage[T1]{fontenc}		% Seleção de códigos de fonte.
\usepackage[utf8]{inputenc}		% Codificação do documento (conversão automática dos acentos)
\usepackage{indentfirst}		% Indent o primeiro parágrafo de cada seção.
\usepackage{color}				% Controle das cores
\usepackage{graphicx}			% Inclusão de gráficos
\usepackage{microtype} 			% para melhorias de justificação
\usepackage{csquotes}
\usepackage{enumitem}		% Controle avançado de listas
\usepackage{amsmath, amssymb, amsthm} % Pacotes matemáticos
\usepackage{float}				% para figuras [H]
\usepackage{multirow}			% para tabelas com células que ocupam mais de uma linha
\usepackage{makecell}			% para tabelas com células quebras mais de uma linha
\usepackage{array}				% para tabelas com células

% ---
% Pacotes adicionais, usados apenas no âmbito do Modelo Canônico do abnteX2
% ---
\usepackage{lipsum}				% para geração de dummy text

% --- Carrega o biblatex com as opções equivalentes ---
\usepackage[
    backend=biber,
    style=abnt,
    backref=true,      % <-- Ativa as referências de volta (substitui o pacote backref)
    maxcitenames=3,  % <-- Substituto para "etalcite=4". Mostra até 3 autores, acima disso usa "et al."
    maxbibnames=99   % <-- Substituto para "etallist=0". Mostra todos os autores na lista final.
]{biblatex}
\addbibresource{REFERENCIAS.bib} % Comando para declarar o arquivo .bib

\usepackage{pgfgantt}
\usepackage{hyperref}  % Moved earlier; load as one of the last packages for compatibility

% --- 
% CONFIGURAÇÕES DE PACOTES
% --- 

% Define bibliography strings after loading biblatex
\DefineBibliographyStrings{portuguese}{%
  backrefpage  = {Citado na página},   % Texto para uma única citação em uma página
  backrefpages = {Citado nas páginas}, % Texto para múltiplas citações em várias páginas
}

% Define custom Gantt link type after loading pgfgantt
\newganttlinktype{straight}{
  \ganttsetstartanchor{on right=1}
  \ganttsetendanchor{on left=0}
  \draw[/pgfgantt/link]
    ([xshift=-0.2pt]\xLeft, \yUpper) --
    (\xRight, \yLower);  % chktex 8  (suppresses false positive on dash in TikZ path)
}


% Informações de dados para CAPA e FOLHA DE ROSTO

\titulo{Projeto de Pesquisa\\ Cisalhamento em Lajes Espessas}
\autor{Lucas Mateus Farias de Barros}
\local{Brasil}
\data{2025}
\instituicao{%
  Universidade de São Paulo
  \par
  Escola Politécnica
  \par
  Programa de Pós-Graduação}
\tipotrabalho{Projeto de Pesquisa (Mestrado)}
% O preambulo deve conter o tipo do trabalho, o objetivo, 
% o nome da instituição e a área de concentração 
\preambulo{Este projeto de pesquisa busca investigar o cisalhamento em lajes com espessura superior a 60 centímetros e o comportamento dessas estruturas. 
Será proposto uma atualização à ABNT NBR 6118, alinhando-a as normas internacionais e garantindo segurança em projetos estruturais complexos.}
% ---

% ---
% Configurações de aparência do PDF final
% ---

% alterando o aspecto da cor azul
\definecolor{blue}{RGB}{41,5,195}

% informações do PDF (after loading hyperref)
\makeatletter
\hypersetup{
     	%pagebackref=true,
		pdftitle={\@title}, 
		pdfauthor={\@author},
    	pdfsubject={\imprimirpreambulo},
	    pdfcreator={LaTeX with abnTeX2},
		pdfkeywords={abnt}{latex}{abntex}{abntex2}{projeto de pesquisa}, 
		colorlinks=true,       		% false: boxed links; true: colored links
    	linkcolor=blue,          	% color of internal links
    	citecolor=blue,        		% color of links to bibliography
    	filecolor=magenta,      		% color of file links
		urlcolor=blue,
		bookmarksdepth=4
}
\makeatother
% --- 

% --- 
% Espaçamentos entre linhas e parágrafos 
% --- 

% O tamanho do parágrafo é dado por:
\setlength{\parindent}{1.3cm}

% Controle do espaçamento entre um parágrafo e outro:
\setlength{\parskip}{0.2cm}  % tente também \onelineskip

% ---
% compila o índice
% ---
\makeindex
% ---