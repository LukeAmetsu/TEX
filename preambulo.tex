% ---
% PACOTES
% ---

% ---
% Pacotes fundamentais
% ---
\usepackage{lmodern}
\usepackage[T1]{fontenc}
\usepackage[utf8]{inputenc}
\usepackage{indentfirst}
\usepackage{graphicx}
\usepackage{microtype}
\usepackage{csquotes}
\usepackage{enumitem}
\usepackage{amsmath, amssymb, amsthm}
\usepackage{float}
\usepackage{multirow}
\usepackage{makecell}
\usepackage{array}

% ---
% Pacotes adicionais
% ---
\usepackage{lipsum}

% --- Carrega o biblatex ---
\usepackage[
    backend=biber,
    style=abnt,
    backref=true,
    maxcitenames=3,
    maxbibnames=99
]{biblatex}
\addbibresource{REFERENCIAS.bib}

% --- Pacotes para o Gantt e diagramas ---
\usepackage{pgfgantt} 
\usepackage{caption}
\usetikzlibrary{calc, shadows, shadings}

% --- Hyperref DEVE ser um dos últimos pacotes a serem carregados ---
\usepackage{hyperref}

% ---
% CONFIGURAÇÕES DE PACOTES
% ---

\DefineBibliographyStrings{portuguese}{%
  backrefpage  = {Citado na página},
  backrefpages = {Citado nas páginas},
}

% --- Informações do Documento ---
\titulo{Projeto de Pesquisa\\ Cisalhamento em Lajes Espessas}
\autor{Lucas Mateus Farias de Barros}
\local{Brasil}
\data{2025}
\instituicao{%
  Universidade de São Paulo
  \par
  Escola Politécnica
  \par
  Programa de Pós-Graduação}
\tipotrabalho{Projeto de Pesquisa (Mestrado)}
\preambulo{Este projeto de pesquisa busca investigar o cisalhamento em lajes com espessura superior a 60 centímetros e o comportamento dessas estruturas. Será proposto uma atualização à ABNT NBR 6118, alinhando-a as normas internacionais e garantindo segurança em projetos estruturais complexos.}

% --- Configurações de aparência do PDF ---
\definecolor{blue}{RGB}{41,5,195}

\makeatletter
\hypersetup{
    pdftitle={\@title},
    pdfauthor={\@author},
    pdfsubject={\imprimirpreambulo},
    pdfcreator={LaTeX with abnTeX2},
    pdfkeywords={abnt, latex, abntex, abntex2, projeto de pesquisa},
    colorlinks=true,
    linkcolor=blue,
    citecolor=blue,
    filecolor=magenta,
    urlcolor=blue,
    bookmarksdepth=4
}
\makeatother

% --- Espaçamentos ---
\setlength{\parindent}{1.3cm}
\setlength{\parskip}{0.2cm}

% --- Compila o índice ---
\makeindex