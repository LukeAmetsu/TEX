% ----
% Início do documento
% ----
\begin{document}

% Seleciona o idioma do documento
\selectlanguage{brazil}

% Retira espaço extra obsoleto entre as frases.
\frenchspacing

% ----------------------------------------------------------
% ELEMENTOS PRÉ-TEXTUAIS
% ----------------------------------------------------------
\imprimircapa
\cleardoublepage
\imprimirfolhaderosto

% --- Sumário ---
\pdfbookmark[0]{\contentsname}{toc}
\tableofcontents*
\cleardoublepage

% ----------------------------------------------------------
% ELEMENTOS TEXTUAIS
% ----------------------------------------------------------
\textual

% ----------------------------------------------------------
% Introdução
% ----------------------------------------------------------
\chapter[Introdução]{Introdução}

O \textbf{concreto simples} é um material não homogêneo, composto de agregados, cimento e água. O concreto destaca-se pela elevada resistência à compressão. Contudo, sua baixa resistência à tração limita severamente seu emprego estrutural. A combinação com barras de aço dá origem ao \textbf{concreto armado}, no qual a aderência entre o concreto e o aço permite resistir eficientemente a esforços de tração e compressão, conferindo ductilidade aos elementos. Um passo adiante, a introdução de forças de compressão por meio de cordoalhas de aço de alta resistência resulta no \textbf{concreto protendido}, que otimiza o uso do material ao controlar a fissuração e as deformações, possibilitando vãos maiores e seções mais esbeltas.\cite{Araujo2023}

O comportamento ao cisalhamento do concreto, especialmente em elementos sem armadura transversal, é um fenômeno complexo. Diferentemente da flexão, governada por teorias bem consolidadas, como Bernoulli-Euler ou Timoshenko, o cisalhamento envolve múltiplos mecanismos de resistência que interagem de forma não linear. Conforme teorias como a \textbf{Teoria do Campo de Compressão Modificada} (\textit{Modified Compression Field Theory -- MCFT}) por \textcite{Vecchio1986} e a \textbf{Teoria da Fissura Crítica de Cisalhamento} (\textit{Critical Shear Crack Theory -- CSCT}) como descrito por \textcite{Muttoni2023}, a resistência ao cisalhamento após a fissuração depende de um conjunto de contribuições, incluindo:

\begin{alineas}
\item a contribuição da zona de concreto comprimido não fissurada;
\item o efeito de pino da armadura longitudinal (\textit{dowel action});
\item o atrito entre as faces da fissura, o engrenamento dos agregados (\textit{aggregate interlock}).
\item a contribuição da armadura transversal, quando presente.
\item resistência residual do concreto tracionado, que é desconsiderada na Teoria do Campo de Compressão Modificada (MCFT).
\end{alineas}

Elementos de maior espessura tendem a desenvolver fissuras mais abertas, o que reduz a eficácia dos mecanismos de transferência de cisalhamento pela interface, como o engrenamento dos agregados. Este fenômeno, que leva à diminuição da resistência específica com o aumento da dimensão do elemento, é conhecido como \textbf{efeito de escala} (\textit{size effect}).

A \textbf{ABNT NBR 6118:2023}, em seu item 19.4, estabelece os critérios para a verificação da resistência ao cisalhamento em lajes. Para lajes sem armadura transversal, o item 19.4.1 introduz o fator $k$, que ajusta a resistência em função da altura útil da laje ($d$) e das condições de ancoragem da armadura longitudinal:

\begin{alineas}
    \item para elementos em que menos de 50\% da armadura inferior está ancorada até o apoio: $k = 1$;
    \item para os demais casos: $k = 1{,}6 - d$, não menor que $1{,}0$, sendo $d$ a altura útil da laje expressa em metros.
\end{alineas}

A origem deste coeficiente é empírica e baseada nos  estudos experimentais conduzidos por \textcite{Hedman1975shear}, que identificaram a redução da resistência nominal ao cisalhamento com o aumento da altura útil da laje ($d$). O fator $k$ visa, portanto, capturar de forma simplificada o efeito de escala. No entanto, por sua natureza empírica, sua validade é restrita a lajes com espessuras usuais, tipicamente inferiores a 60 cm, conforme orienta o \textcite{IBRACON2020}.

Com o aumento de obras de infraestrutura, como as estações de metrô, torna-se comum lajes maciças com espessura superior a 60 centímetros. Como orientado pelo \textcite{IBRACON2020}, nesse caso é comum o uso do \textbf{Eurocode 2}, que incorpora o efeito de escala de forma mais explícita. A seção 6.2.2(1) do Eurocode 2 define o fator $k$ como:
\begin{equation}
    k = 1+ \sqrt{\frac{200}{d}} \leq 2,0
\end{equation}
onde $d$ é a altura útil da laje em milímetros. Essa expressão reduz a resistência ao cisalhamento em lajes espessas, refletindo o efeito de escala observado experimentalmente.

Similarmente, o \textbf{ACI 318:2019}\cite{ACI318:2019} introduziu um fator de modificação para o efeito de escala, $\lambda_s$, definido por:
\begin{equation}
\lambda_s = \sqrt{\frac{2}{1+\frac{d}{10}}}\leq 1,0
\end{equation}
com $d$ é a altura útil em polegadas. Esse fator $\lambda_S$ busca reduzir a resistência ao cisalhamento de lajes espessas, entretanto pesquisadores como \textcite{Marquesi2021} criticam a calibração do fator, que pode reduzir a resistência ao cisalhamento em lajes usuais em até 40\% e em fundações diretas espessas em até 2,5 vezes, apontando que ela foi elaborada com um banco de dados que misturou indiscriminadamente ensaios de lajes e vigas de pequenas dimensões.

Essa abordagem metodológica é questionável, pois vigas e lajes podem apresentar modos de ruptura e comportamentos estruturais distintos. A inclusão de vigas de pequenas dimensões pode ter introduzido um viés, distorcendo a formulação destinada a lajes espessas.

Esta pesquisa justifica-se como uma contribuição para o projeto de lajes espessas e possível atualização da NBR 6118 que alinhe a prática nacional às normas internacionais, com o objetivo de garantir a segurança estrutural e promover a otimização de projetos, evitando tanto o risco de rupturas frágeis quanto o superdimensionamento.

\chapter{Objetivos}
\section{Objetivo Geral}

Investigar o comportamento ao cisalhamento de lajes espessas com o intuito de propor ajustes à formulação da ABNT NBR 6118, assegurando sua aplicabilidade em estruturas de grande porte e complexidade.

\section{Objetivos Específicos}
\begin{alineas}
    \item Realizar uma revisão bibliográfica e normativa sobre o cisalhamento em elementos de concreto sem armadura transversal, com ênfase nos modelos mecânicos que fundamentam o efeito de escala;
    \item Avaliar o comportamento e os modos de ruptura em Estados Limites Últimos (ELU) a partir de bancos de dados experimentais consolidados, como o de \textcite{Kuchma2019};
    \item Comparar o desempenho das formulações da \textcite{NBR6118:2023}, \textcite{CEN2004}, \textcite{ACI318:2019} e \textcite{FIB:2020} na previsão da resistência de lajes espessas;
    \item Desenvolver e propor uma extensão à formulação da ABNT NBR 6118 que incorpore o efeito de escala de forma mecanicamente consistente, visando sua potencial inclusão em futuras revisões da norma;
    \item Calibrar os coeficientes parciais de segurança da formulação proposta via Teoria da Confiabilidade Estrutural, para garantir um índice de confiabilidade ($\beta$) compatível com os alvos normativos para modos de ruptura frágeis.
\end{alineas}

\chapter{Hipótese de Pesquisa}
Com base na literatura recente e nas discussões normativas internacionais, formulam-se as seguintes hipóteses:

\section{Hipótese 1}
A formulação da ABNT NBR 6118:2023 superestima a resistência ao cisalhamento de lajes com espessura superior a 60 cm por não incorporar adequadamente o efeito de escala, resultando em previsões não conservadoras quando comparadas a resultados experimentais.
\section{Hipótese 2}
A variabilidade estatística da resistência ao cisalhamento em lajes espessas, quando normalizada por um modelo teórico robusto, difere significativamente daquela observada em lajes esbeltas, exigindo uma calibração específica dos coeficientes de segurança para manter um índice de confiabilidade consistente.
\section{Hipótese 3}
A calibração da equação e da formulação da norma brasileira, considerando bancos de dados de testes experimentais específicos em lajes de grande espessura, permitirá evitar o uso desnecessário de armadura transversal.

\chapter{Síntese bibliográfica}
A resistência ao cisalhamento em elementos de concreto sem armadura transversal é um tema central na engenharia estrutural. O desenvolvimento de modelos de cálculo para este fenômeno remonta ao início do século XX com a treliça de Mörsch \cite{Morsch1909}, que propôs um modelo simplificado para entender o comportamento do concreto ao cisalhamento.

\section{Modelo Clássico - A Treliça de Mörsch}
Esse modelo, também conhecido como analogia da treliça clássica de Ritter-Mörsch, idealiza o comportamento de uma viga fissurada como uma treliça isostática, onde:

\begin{alineas}
    \item os tirantes (armadura longitudinal e transversal) resistem às forças de tração;
    \item as bielas de concreto comprimido resistem às forças de compressão.
\end{alineas}

Ensaios conduzidos por \textcite{Leonhardt1964} demonstraram que a resistência observada em vigas com baixa taxa de armadura transversal era superior à prevista pela treliça clássica. Isso o levou a propor a inclusão de uma parcela de resistência atribuída ao concreto ($V_c$), reconhecendo a capacidade do concreto fissurado de transmitir cisalhamento por meio de mecanismos como o engrenamento de agregados. A introdução do componente $V_c$ foi um avanço significativo, mas a sua formulação exata permanece um ponto de debate, dada a complexidade do fenômeno e as limitações dos ensaios experimentais da época.


\section{Modelos Modernos}

Novas demandas de infraestruturas impulsionaram o desenvolvimento de modelos mais sofisticados para representar o comportamento do concreto ao cisalhamento. 

\subsection{A Teoria do Campo de Compressão Modificada (MCFT)}

Proposta por \textcite{Vecchio1986}, a Teoria do Campo de Compressão Modificado (MCFT), trata o concreto fissurado como um material que a resistência ao cisalhamento é influenciada por múltiplos fatores, que dependem do estado de deformação. Seus principais conceitos incluem:

\begin{alineas}
    \item Análise baseada em tensões e deformações médias, aplicadas a uma região representativa do elemento;
    \item Consideração do "amolecimento" do concreto (redução da resistência à compressão devido a deformações de tração transversais);
    \item  Integração de múltiplos mecanismos de resistência, como a contribuição do concreto, o efeito de pino e o atrito na fissura.
\end{alineas}

\subsection{A Teoria da Fissura Crítica de Cisalhamento (CSCT)}

Desenvolvida por \textcite{Muttoni2017}, a Teoria da Fissura Crítica de Cisalhamento (CSCT) propõe que a ruptura por cisalhamento vem da formação de uma fissura crítica. A capacidade resistente ao cisalhamento é inversamente proporcional à largura dessa fissura, que, por sua vez, depende da deformação na armadura longitudinal. Este modelo fornece uma explicação racional para o efeito de escala e influenciou diretamente as formulações do fib Model Code 2010 e da segunda geração do Eurocode 2.

\section{Normas e o Efeito de Escala}
A contínua pesquisa na área de cisalhamento tem levado a uma revisão crítica dos modelos presentes nas normas técnicas, tanto no Brasil quanto no exterior. Análises recentes, utilizando extensos bancos de dados experimentais, têm quantificado as imprecisões de modelos estabelecidos e guiado o desenvolvimento de novas formulações.

\subsection{As Mudanças e Críticas ao Modelo ACI 318}
O modelo de cálculo do ACI 318-14 sofria com críticas devido a sua alta variabilidade e por não considerar o efeito de escala. A revisão de 2019 introduziu o fator de escala ($\lambda_s$) definido por:

\begin{equation}
    \lambda_s = \sqrt{\frac{2}{1+\frac{d}{10}}}\leq 1,0
\end{equation}

Essa mudança foi baseada em extensos bancos de dados desenvolvidos em colaboração entre o comitê ACI-ASCE 445 e o comitê alemão DAfStb. No entanto, como aponta \textcite{Marquesi2021}, a calibração dessa fórmula combinou ensaios de vigas e lajes, o que aumentou a variabilidade dos resultados.
A aplicação do novo modelo reduziu drasticamente a resistência calculada para lajes usuais (e.g., uma redução de 33\% para uma laje de 18 cm. Para uma sapata de 75 cm, a redução chegou a quase 45\%), levantando questionamentos sobre a segurança de inúmeras estruturas existentes projetadas com critérios anteriores e que não apresentam patologias.

\subsection{Análise do Erro de Modelo da NBR 6118:2014 por Barros et al.}

O estudo de \textcite{Barros2021} analisou o mesmo banco de dados utilizado por \textcite{Kuchma2019}, avaliou o desempenho da NBR 6118:2014. Os resultados obtidos indicaram que a norma brasileira apresentava um coeficiente de variação (COV) de 0,291 e um desvio padrão de 0.301, sugerindo um resultado inadequado e uma tendência a superestimar a resistência de lajes espessas. Entretanto, ao se analisar a equação \ref{Vrd1mod} proposta por \textcite{Barros2021} e os gráficos de dispersão apresentados, percebe-se a similaridade com os dados apresentados por \textcite{Kuchma2019}, sugerindo que houve a combinação indiscriminada de ensaios de vigas e lajes, o que pode ter influenciado negativamente na precisão dos resultados, especialmente levando em conta que vigas sem armadura de cisalhamento não são permitidas pela NBR 6118.
\begin{equation}
    V_{Rd1} = 1.35 \times \left(\frac{0.6}{d}\right)^{0.4} \times (\rho)^\frac{1}{3} \times 0.25f_{ctd}\times d \times b_w
    \label{Vrd1mod}
\end{equation}

\section{Lacunas na Literatura e Necessidade da Pesquisa}
Apesar dos avanços significativos na compreensão do comportamento ao cisalhamento, a maioria dos estudos concentra-se em elementos esbeltos. A extrapolação de fórmulas propostas para esses elementos esbeltos em lajes espessas, comuns em infraestrutura moderna, é inadequada e potencialmente insegura. Dessa maneira, torna-se imperativo a análise específica de lajes espessas e a confiabilidade desse resultado, de maneira que a recomendações de projeto da NBR 6118 conduzam a projetos seguros e eficientes para as obras de infraestrutura no Brasil.

\chapter{Metodologia}
A pesquisa seguirá a seguinte metodologia:

\begin{alineas}
    \item Estudo aprofundado das bases conceituais das disposições de cisalhamento nas normas ABNT NBR 6118, Eurocode 2, ACI 318 e fib Model Code 2010.
    \item Análise crítica das teorias fundamentais, como a MCFT e a CSCT, para embasar o desenvolvimento de uma formulação mecanicamente consistente.
    \item Utilização do banco de dados do ACI-ASCE 445 e do Comitê Alemão para Concreto Estrutural (\textit{DAbStb}), com foco em ensaios de lajes de espessura superior a 60 cm. O banco de dados pode ser obtido após demanda ao comitê ACI--DAbStb disponível em (https://dafstb.de/aci-dafstb.html). O banco de dados será complementado com buscas em bases acadêmicas para garantir robustez estatística.
    \item Análise estatística da razão entre os resultados experimentais e as previsões de cada norma, calculando-se a média (acurácia), o desvio padrão e o coeficiente de variação (precisão).
    \item Desenvolvimento de uma formulação candidata para incorporação à norma brasileira, com aplicabilidade validada para elementos espessos.
    \item Calibração dos coeficientes parciais de segurança da proposta via Teoria da Confiabilidade Estrutural, definindo um estado limite de falha e variáveis aleatórias para atingir um índice de confiabilidade ($\beta$) alvo, consistente com o preconizado pela ABNT NBR 6118 para rupturas frágeis.
    \item A calibração da fórmula será condicionada a não alterar significativamente os resultados para lajes esbeltas, preservando o desempenho histórico da NBR 6118 nesse domínio.
    \item Avaliação da robustez da formulação proposta frente à variabilidade de parâmetros construtivos, como resistência do concreto e posicionamento da armadura.
\end{alineas}

Para a execução, serão necessárias as seguintes infraestruturas:
\begin{alineas}
    \item Acesso a bases de dados acadêmicas (Scopus, Web of Science, etc.).
    \item Software estatístico ou linguagens de programação (e.g., Python, R) para análise de dados.
\end{alineas}

\chapter{Cronograma}

O cronograma previsto para a execução do projeto de pesquisa é:
\begin{alineas}
    \item Meses 1 a 4 - Revisão bibliográfica aprofundada.
    \item Meses 4 a 8 - Coleta, tratamento e análise do banco de dados.
    \item Meses 8 a 12 - Preparação da documentação para a qualificação.
    \item Meses 12 a 14 - Análise preliminar e ajuste da formulação proposta.
    \item Meses 14 a 18 - Validação da formulação proposta frente aos dados experimentais.
    \item Meses 18 a 20 - Análise de confiabilidade e calibração dos coeficientes.
    \item Meses 20 a 24 - Redação final da dissertação.
\end{alineas}

\begin{figure}[htb!]
\label{cronograma}
\begin{center}
\begin{ganttchart}[
    canvas/.append style={fill=none, draw=black!5, line width=.75pt},
    vgrid,
    hgrid,
    x unit=0.51cm,
    y unit title=1cm,
    y unit chart=1.25cm,
    title/.style={draw=none, fill=none},
    title label font=\bfseries\footnotesize,
    title label node/.append style={below=1pt},
    include title in canvas=false,
    bar label font=\mdseries \small \color{black!70},
    link/.style={-latex, line width=1.5pt},
    bar label node/.append style={align=left, text width=3cm}
]{1}{25}
% ==== TÍTULOS ====
\gantttitle[
    title label node/.append style={below left=1pt and -7pt}
    ]{Meses:\quad1}{1}
\gantttitlelist{2,...,24}{1} \ganttnewline
% ==== ATIVIDADES ====
\ganttbar[bar/.append style={fill=teal!50}]{Revisão \\bibliográfica}{1}{4}\ganttnewline
\ganttbar[bar/.append style={fill=orange!50}]{Análise do banco \\de dados}{4}{8}\ganttnewline
\ganttbar[bar/.append style={fill=cyan!50}]{Preparação para a qualificação}{8}{12}
\ganttnewline
\ganttmilestone[milestone/.append style={fill=red!80!black}]{Qualificação}{12} \ganttnewline
\ganttbar[bar/.append style={fill=purple!40}]{Análise e ajuste da formulação}{12}{14}\ganttnewline
\ganttbar[bar/.append style={fill=lime!50!gray}]{Validação da formulação}{14}{18}\ganttnewline
\ganttbar[bar/.append style={fill=blue!40}]{Análise de confiabilidade}{18}{20}\ganttnewline
\ganttbar[bar/.append style={fill=magenta!40}]{Redação final da dissertação.}{20}{24} \ganttnewline
\ganttmilestone[milestone/.append style={fill=green!60!black}]{Defesa Final}{24}
\end{ganttchart}
\end{center}
\caption{Cronograma proposto da pesquisa}
\end{figure}

\chapter[Resultados Esperados]{Resultados Esperados}
Essa pesquisa busca abordar uma lacuna crítica na engenharia estrutural brasileira: a verificação do cisalhamento em lajes espessas ($\geq$ 60 centímetros). A extrapolação das fórmulas da ABNT NBR 6118:2023 para tais estruturas pode comprometer a segurança estrutural, uma preocupação crescente diante da execução de obras de infraestrutura complexas. A necessidade de alinhar a prática nacional às normas internacionais, que já incorporam o efeito de escala, justifica a presente investigação.

O principal resultado esperado é uma formulação para a resistência ao cisalhamento em lajes espessas, calibrada e validada, que possa ser incorporada na ABNT NBR 6118. Tal contribuição visa impactar a engenharia brasileira em três frentes principais:

\begin{alineas}
   \item \textbf{Atualização normativa:} Alinhar a norma brasileira com o estado da arte internacional, preenchendo uma lacuna técnica importante.
    \item \textbf{Segurança e eficiência:} Fornecer aos engenheiros projetistas uma ferramenta de verificação mais segura e precisa, evitando tanto o superdimensionamento quanto o risco de rupturas frágeis.
    \item \textbf{Conhecimento técnico:} Aprofundar o entendimento sobre o efeito de escala e o comportamento de estruturas de concreto em cenários ainda não cobertos adequadamente pela literatura técnica nacional.
\end{alineas}

% ----------------------------------------------------------
% ELEMENTOS PÓS-TEXTUAIS
% ----------------------------------------------------------
\postextual

% ----------------------------------------------------------
% Referências bibliográficas
% ----------------------------------------------------------

\printbibliography

\end{document}